\documentclass[12pt]{article}

\usepackage{amsmath}
\usepackage{amssymb}
\usepackage{graphicx}
\usepackage{hyperref}

\title{Basic Postulates of Quantum Mechanics}
\author{Author Name}
\date{\today}

\begin{document}

\maketitle

\begin{abstract}
This document outlines the basic postulates of quantum mechanics.
\end{abstract}

\section{Introduction}
Quantum mechanics is a fundamental theory in physics that describes the physical properties of nature at the scale of atoms and subatomic particles.

\section{Postulate 1: State Vector}
The state of a quantum mechanical system is completely specified by a function $\psi$ that depends on the coordinates of the particle and on time. This function, called the wave function or state vector, contains all the information about the system.

\section{Postulate 2: Observables and Operators}
In quantum mechanics, every observable quantity is associated with a linear, Hermitian operator. The possible outcomes of measuring an observable are the eigenvalues of the corresponding operator.

\section{Postulate 3: Measurement}
The only possible result of the measurement of a physical quantity is one of the eigenvalues of the operator corresponding to that quantity. The probability of obtaining a particular eigenvalue is given by the square of the absolute value of the projection of the state vector onto the eigenvector corresponding to that eigenvalue.

\section{Postulate 4: Time Evolution}
The time evolution of the state of a quantum mechanical system is governed by the Schrödinger equation:
\[
i\hbar \frac{\partial \psi}{\partial t} = \hat{H} \psi
\]
where $\hat{H}$ is the Hamiltonian operator of the system.

\section{Conclusion}
These postulates form the foundation of quantum mechanics and provide a framework for understanding the behavior of quantum systems.

\end{document}